\documentclass[12pt]{article}
\usepackage[utf8]{inputenc}
\usepackage{amsthm}
\usepackage{geometry}
\usepackage{amssymb}
\usepackage{enumerate}
\geometry{margin=2cm}

% \setlength\parindent{0pt}
% \setlength{\parskip}{\baselineskip}

\title{MLCS - Homework 1}
\author{Michele Conti \\ \texttt{1599133}}
\date{}


\theoremstyle{definition}
% \newtheorem*{exercise}{Exercise}
\newtheorem{exerciseinner}{Exercise}
\newenvironment{exercise}[1]{%
  \renewcommand\theexerciseinner{#1}%
  \exerciseinner
}{\endexerciseinner}

\renewcommand*{\proofname}{Solution}

\begin{document}
\maketitle

\section{Group 1}

\begin{exercise}{2}
Show that the following sentence is unsatisfiable, where $S$ is any formula with two free variables: $\exists x \forall y (S(x, y) \leftrightarrow \neg S(y, y))$.
\end{exercise}
\begin{proof}
Given any choice of $x$, it is sufficient to pick $y = x$, in which case we would get 
\[
    S(x, x) \leftrightarrow \neg S(x, x).
\]
Since this is obviously a contradiction, it means that it exists $y$ such that the formula $S$ is false, meaning that it is unsatisfiable.
\end{proof}

\begin{exercise}{4}
Is the following formula logically valid for any formula $F$ and any term $t$?
\[
    \forall x F(x) \to F(t).
\]
If not, give an example of a formula $F$, a structure $\mathfrak{A}$ and an assignment $\alpha$ witnessing this fact.
\end{exercise}

\begin{proof}
If we analyze the premise of the implication, we notice that we only have two cases:
\begin{enumerate}
    \item $F(x)$ is true for every $x$ (i.e., the premise is true).
    \item It exists $x$ such that $F(x)$ is false (i.e., the premise is false). 
\end{enumerate}
In case 1, we have that for every choice of $x$ the formula $F$ is true, meaning that it is also true for the given term $t$. Therefore, the implication is true, because we are considering an implication between a two true statements.

In case 2, the premise of the implication is false, therefore we can conclude that the whole implication is true.
\end{proof}

\begin{exercise}{9}
In the language $\mathcal{L} = \{<\}$ of \textbf{DLO}, write a sentence that distinguishes $(\textbf{N}, <)$ from $(\textbf{Q}, <)$ i.e., that is true in one structure but not in the other.
\end{exercise}
\begin{proof}
The following sentence is true in $(\textbf{N}, <)$, but not in $(\textbf{Q}, <)$:
\[
    \forall x \exists y (\neg(x = y) \to y < x).
\]

This represents the fact that $\mathbb{Q}$ has no left endpoint, while $\mathbb{N}$ has $0$ (or $1$, depending on whether we consider the set to have or not the element $0$), which is lower than any other element.
\end{proof}

\section{Group 2}

\begin{exercise}{3}
Is the structure $\mathcal{Q} = (\mathbb{Q}, +, \times, 0, 1)$ a substructure $\mathcal{R} = (\mathbb{R}, +, \times, 0, 1)$? Is it an elementary substructure?
\end{exercise}
\begin{proof}
My assumption is that $\mathcal{Q}$ is a substructure of $\mathcal{R}$, but that it is not an elementary substructure. Let's prove both points.

To prove that $\mathfrak{B}$ is a substructure of $\mathfrak{A}$, we need to check if
\begin{enumerate}
    \item $B \subseteq A$.
    \item For every constant symbol $c$, $c^{\mathfrak{A}} = c^{\mathfrak{B}}$.
    \item Every relation $R^{\mathfrak{B}}$ (resp. function $f^{\mathfrak{B}}$) is the restriction of $R^{\mathfrak{A}}$ (resp. $f^{\mathfrak{A}}$) to B.
\end{enumerate}
For the first point, there is nothing to prove. The second point is trivially true, since we only have two constants in both structures (i.e., $0$ and $1$) and they correspond. The third point (?)

To prove that the substructure $\mathfrak{B}$ is not elementary, we can consider the formula
\[
    \forall x \exists y ((x > 0) \to (y \times y = x)).
\]
This is clearly true for $\mathcal{R}$, but it is not for $\mathcal{Q}$. In fact, we can consider $x = 2$ and it doesn't exist any $y$ such that $y \times y = 2$.
\end{proof}

\begin{exercise}{4}
Prove that the following structures are not isomorphic:
\begin{enumerate}[\quad{(}1{)}]
\item ($\mathbb{N}$ +, $\times$, $0$, $1$, $<$) and ($\mathbb{Q}$, $+$, $\times$, $0$, $1$, $<$).
\item ($\mathbb{N}$, $<$) and ($\mathbb{Z}$, $<$).
\item ($\mathbb{Q}$, $<$) and ($\mathbb{R}$, $<$).
\end{enumerate}
(Hint: in some cases you can use the fact that if $\mathfrak{A}$ and $\mathfrak{B}$ are isomorphic then they satisfy the same sentences).
\end{exercise}
\begin{proof}
Let's consider case by case.

To prove that ($\mathbb{N}$ +, $\times$, $0$, $1$, $<$) and ($\mathbb{Q}$, $+$, $\times$, $0$, $1$, $<$) are not isomorphic, we can consider the sentence that expresses the existence of additive inverse:
\[
    \forall x \exists y (x + y = 0)
\]
is satisfied in ($\mathbb{Q}$, $+$, $\times$, $0$, $1$, $<$), but not in ($\mathbb{N}$ +, $\times$, $0$, $1$, $<$). We know that two structures are isomorphic if and only if they satisfy the same sentences, therefore we can conclude that these two are not.

To prove that ($\mathbb{N}$, $<$) and ($\mathbb{Z}$, $<$) are not isomorphic, we can proceed in a similar way, using the sentence expressing the existence of the left-end point:
\[
    \dots
\]
\end{proof}


\end{document}
