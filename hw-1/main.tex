\documentclass[12pt]{article}
\usepackage[utf8]{inputenc}
\usepackage{amsthm,xpatch}
\usepackage{amsmath}

\usepackage{geometry}
\usepackage{amssymb}
\usepackage{enumerate}
\geometry{margin=2cm}

%\setlength\parindent{0pt}
% \setlength{\parskip}{\baselineskip}

\title{MLCS - Homework 1}
\author{Michele Conti \\ \texttt{1599133}}
\date{}


\theoremstyle{definition}
% \newtheorem*{exercise}{Exercise}
\newtheorem{exerciseinner}{Exercise}
\newenvironment{exercise}[1]{%
  \renewcommand\theexerciseinner{#1}%
  \exerciseinner
}{\endexerciseinner}

\renewcommand*{\proofname}{Solution}

\makeatletter
\newcounter{proofpart}
\xpretocmd{\proof}{\setcounter{proofpart}{0}}{}{}
\newcommand{\proofpart}[1]{%
	\par
	\addvspace{\medskipamount}%
	\stepcounter{proofpart}%
	\noindent\emph{Part \theproofpart: #1}\par\nobreak\smallskip
	\@afterheading
}
\makeatother




\begin{document}
\maketitle

\section{Group 1}

\begin{exercise}{2}
Show that the following sentence is unsatisfiable, where $S$ is any formula with two free variables: $\exists x \forall y (S(x, y) \leftrightarrow \neg S(y, y))$.
\end{exercise}
\begin{proof}
Given any choice of $x$, it is sufficient to pick $y = x$, in which case we would get 
\[
    S(x, x) \leftrightarrow \neg S(x, x).
\]
Since this is obviously a contradiction, it means that it exists $y$ such that the formula $S$ is false, meaning that it is unsatisfiable.
\end{proof}

\begin{exercise}{4}
Is the following formula logically valid for any formula $F$ and any term $t$?
\[
    \forall x F(x) \to F(t).
\]
If not, give an example of a formula $F$, a structure $\mathfrak{A}$ and an assignment $\alpha$ witnessing this fact.
\end{exercise}

\begin{proof}
If we analyze the premise of the implication, we notice that we only have two cases:
\begin{enumerate}
    \item $F(x)$ is true for every $x$ (i.e., the premise is true).
    \item It exists $x$ such that $F(x)$ is false (i.e., the premise is false). 
\end{enumerate}
In case 1, we have that for every choice of $x$ the formula $F$ is true, meaning that it is also true for the given term $t$. Therefore, the implication is true, because we are considering an implication between a two true statements.

In case 2, the premise of the implication is false, therefore we can conclude that the whole implication is true.
\end{proof}

\begin{exercise}{9}
In the language $\mathcal{L} = \{<\}$ of \textbf{DLO}, write a sentence that distinguishes $(\textbf{N}, <)$ from $(\textbf{Q}, <)$ i.e., that is true in one structure but not in the other.
\end{exercise}
\begin{proof}
The following sentence is true in $(\textbf{N}, <)$, but not in $(\textbf{Q}, <)$:
\[
    \forall x \exists y (\neg(x = y) \to y < x).
\]

This represents the fact that $\mathbb{Q}$ has no left endpoint, while $\mathbb{N}$ has $0$ (or $1$, depending on whether we consider the set to have or not the element $0$), which is lower than any other element.
\end{proof}

\section{Group 2}
\begin{exercise}{2}
Let $\mathfrak{A}_1 = (\textbf{N}, +, 0)$ and $\mathfrak{A}_2 = (2\textbf{N}, +, 0)$ be two substructures for the language $\mathcal{L} = \{f, c\}$ where $f$ is a function symbol of arity $2$ and $c$ is a constant symbol and $2 \textbf{N}$ denotes the set of even natural numbers. $\mathfrak{A}_1$ and $\mathfrak{A}_2$ interpret $f$ as the sum of their domains and $c$ as $0$. Indicate whether the following are true or false, giving a short justification of your answer.
\begin{enumerate}[\quad{(}1{)}]
	\item $\mathfrak{A}_2$ is a substructure of $\mathfrak{A}_1$.
	\item $\mathfrak{A}_1$ and $\mathfrak{A}_2$ are isomorphic.
	\item $\mathfrak{A}_1$ and $\mathfrak{A}_2$ satisfy the same sentences in $\mathcal{L}$.
	\item If $\mathfrak{A}_1 \models \exists x F(x)[\alpha]$ for an assignment $\alpha$ in $A_2$, then there exists $a \in A_1$ such that $\mathfrak{A}_2 \models \exists x F(x)[\alpha \binom{a}{x}]$.
	\item If $E$ is a sentence of the form $\forall x F(x)$ with $F(x)$ a quantifier-free formula then: \\ If $\mathfrak{A}_1 \models E$ then $\mathfrak{A}_2 \models E$.
	\item If $E$ is a sentence of the form $\exists x F(x)$ with $F(x)$ a quantifier-free formula, then: \\ If $\mathfrak{A}_1 \models E$ then $\mathfrak{A}_2 \models E$.
\end{enumerate}
\end{exercise}
\begin{proof}
Let's consider each point separately.
\proofpart{\textbf{True}.}
Recalling the definition of substructure, we need to prove that:
\begin{enumerate}
	\item $B \subseteq A$.
	\item For every constant symbol $c$, $c^{\mathfrak{A}} = c^{\mathfrak{B}}$.
	\item Every relation $R^{\mathfrak{B}}$ (resp. function $f^{\mathfrak{B}}$) is the restriction of $R^{\mathfrak{A}}$ (resp. $f^{\mathfrak{A}}$) to B.
\end{enumerate}

The first point, $2 \textbf{N} \subseteq \textbf{N}$, is trivially true. As for the second point, the only constant both structures have is $0$, and therefore this is also true. Finally, to prove that every relation (resp. function) in $\mathfrak{A}_2$ is the restriction of the corresponding relation (reps. function) in $\mathfrak{A}_1$, we observe that we only have the operator $+$ in both structures. Now, if we sum two even numbers we still get an even number, and the sum of natural numbers is still a natural number. Therefore, we can conclude that relations or functions in $\mathfrak{A}_2$ are the restrictions of the same relations or functions in $\mathfrak{A}_1$.

\proofpart{\textbf{True}.}
Recalling the definition of isomorphism, we need to prove that it exists a bijection $h \colon \mathfrak{A}_1 \longrightarrow \mathfrak{A}_2$ such that:
\begin{enumerate}
	\item For each $n$-ary relation $R$ in the language, for each $(a_1, \dots, a_n) \in A^n_1$,
	\[
		\mathfrak{A_1} \models R(x_1, \dots, x_n)[a_1, \dots, a_n] \leftrightarrow \mathfrak{A_2} \models R(x_1, \dots, x_n)[h(a_1), \dots, h(a_n)]
	\]
	i.e.
	\[
		(a_1, \dots, a_n) \in R^{\mathfrak{A_1}} \leftrightarrow (h(a_1), \dots, h(a_n)) \in R^{\mathfrak{A_2}}.
	\]
	\item For each constant symbol $c$ in the language,
	\[
		h(c^{\mathfrak{A_1}}) = c^{\mathfrak{A_2}},
	\]
	\item For each $n$-ary function symbol $f$ in the language, for every $(a_1, \dots, a_n) \in A_1^n$,
	\[
		h(f^{\mathfrak{A1}}(a_1, \dots, a_n)) = f^{\mathfrak{A_2}}(h(a_1), \dots, h(a_n)).
	\]
\end{enumerate}
Let's consider the function that maps each number to its double:

\begin{align*}
	h \colon \mathbb{N} & \longrightarrow 2\mathbb{N} \\
	n & \longmapsto n + n.
\end{align*}
Now, $h$ is obviously bijective, and we can separately prove each point of the definition for this map.

For the first point, there is nothing to prove, since neither $\mathfrak{A_1}$ or $\mathfrak{A_2}$ have any relation in the language.

For the second point, we only have constant $0$ in $\mathfrak{A_1}$ (i.e., $0^{\mathfrak{A_1}}$), so we can manually check the application of $h$ on the constant:

\[
	h(0^{\mathfrak{A_1}}) = 0^{\mathfrak{A_1}} + 0^{\mathfrak{A_1}} = 0^{\mathfrak{A_2}}.
\]

Finally, for the last point, since both structures only have the function symbol $+$, we can write:

\begin{align*}
	h(f^{\mathfrak{A1}}(a_1, \dots, a_n)) & = h(a_1 + \dots + a_n) = \\
	& = (a_1 + \dots + a_n) + (a_1 + \dots + a_n) = \\
	& = (a_1 + a_1) + \dots + (a_n + a_n) = \\
	& = h(a_1) + \dots + h(a_n) = \\
	& = f^{\mathfrak{A_2}}(h(a_1), \dots, h(a_n)).
\end{align*}

\proofpart{\textbf{True}.}
From the previous part we know that $\mathfrak{A_1} \simeq \mathfrak{A_2}$. Moreover, we know that if two structures are isomorphic, then they satisfy the same of sentences, which concludes this part.

\proofpart{}
$\dots$

\proofpart{\textbf{True}.}
Consequence of part 3.
\proofpart{\textbf{True}.}
Consequence of part 3.
\end{proof}

\begin{exercise}{3}
Is the structure $\mathcal{Q} = (\mathbb{Q}, +, \times, 0, 1)$ a substructure $\mathcal{R} = (\mathbb{R}, +, \times, 0, 1)$? Is it an elementary substructure?
\end{exercise}
\begin{proof}
My assumption is that $\mathcal{Q}$ is a substructure of $\mathcal{R}$, but that it is not an elementary substructure. Let's prove both points.

\proofpart{$\mathfrak{B}$ is a substructure of $\mathfrak{A}$.}
Recalling the definition of substructure, we need to prove that:
\begin{enumerate}
    \item $B \subseteq A$.
    \item For every constant symbol $c$, $c^{\mathfrak{A}} = c^{\mathfrak{B}}$.
    \item Every relation $R^{\mathfrak{B}}$ (resp. function $f^{\mathfrak{B}}$) is the restriction of $R^{\mathfrak{A}}$ (resp. $f^{\mathfrak{A}}$) to B.
\end{enumerate}
For the first point, there is nothing to prove.

The second point is trivially true, since we only have two constants in both structures (i.e., $0$ and $1$) and they correspond.

For the last point, since both structures only have the sum and multiplication operations in the language, we need to check if the sum and multiplication operations of the first structure $\mathcal{Q}$ (to which we'll refer to respectively with $+^{\mathcal{Q}}$ and $\times^{\mathcal{Q}}$) are the restrictions of the sum and multiplication operations of the second structure $\mathcal{R}$ (to which we'll refer to respectively with $+^{\mathcal{R}}$ and $\times^{\mathcal{R}}$).

\proofpart{Substructure $\mathfrak{B}$ of $\mathfrak{A}$ is not elementary.}
To prove that the substructure $\mathfrak{B}$ is not elementary, we can consider the formula
\[
    \forall x \exists y ((x > 0) \to (y \times y = x)).
\]
This is clearly true for $\mathcal{R}$, but it is not for $\mathcal{Q}$. In fact, we can consider $x = 2$ and it doesn't exist any $y$ such that $y \times y = 2$.
\end{proof}

\begin{exercise}{4}
Prove that the following structures are not isomorphic:
\begin{enumerate}[\quad{(}1{)}]
\item ($\mathbb{N}$ +, $\times$, $0$, $1$, $<$) and ($\mathbb{Q}$, $+$, $\times$, $0$, $1$, $<$).
\item ($\mathbb{N}$, $<$) and ($\mathbb{Z}$, $<$).
\item ($\mathbb{Q}$, $<$) and ($\mathbb{R}$, $<$).
\end{enumerate}
(Hint: in some cases you can use the fact that if $\mathfrak{A}$ and $\mathfrak{B}$ are isomorphic then they satisfy the same sentences).
\end{exercise}
\begin{proof}
Let's consider each point separately.
\proofpart{}
To prove that ($\mathbb{N}$ +, $\times$, $0$, $1$, $<$) and ($\mathbb{Q}$, $+$, $\times$, $0$, $1$, $<$) are not isomorphic, we can consider the sentence that expresses the existence of additive inverse:
\[
    \forall x \exists y (x + y = 0).
\]
This formula is satisfied in ($\mathbb{Q}$, $+$, $\times$, $0$, $1$, $<$), but not in ($\mathbb{N}$ +, $\times$, $0$, $1$, $<$). As for the hint, we know that if two structures are isomorphic, then they satisfy the same sentences. Therefore, we can conclude that these two are not.

\proofpart{}
To prove that ($\mathbb{N}$, $<$) and ($\mathbb{Z}$, $<$) are not isomorphic, we can proceed in a similar way, using the sentence expressing the existence of the left-end point:
\[
	\exists x \forall y (\neg (x = y) \to (x < y)).
\]
This sentence is true in ($\mathbb{N}$, $<$) (i.e., $0$ is the left-end point), but it is false in ($\mathbb{Q}$, $<$).

\proofpart{}
Finally, to prove that ($\mathbb{Q}$, $<$) and ($\mathbb{R}$, $<$) are not isomorphic, we can simply affirm that those structures are not isomorphic for cardinality reasons.
\end{proof}
\end{document}
