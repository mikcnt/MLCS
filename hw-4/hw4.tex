\documentclass[12pt]{article}
\usepackage{graphicx}
\graphicspath{ {./images/} }
\usepackage[utf8]{inputenc}
\usepackage{amsthm,xpatch}
\usepackage{amsmath}
\usepackage{tikz} 
\usepackage{geometry}
\usepackage{amssymb}
\usepackage{enumerate}
\usepackage{float}
\usepackage[english]{babel}
\usetikzlibrary{arrows, automata, calc, positioning}
\geometry{margin=2cm}

\title{MLCS - Homework 4}
\author{Michele Conti \\ \texttt{1599133}}
\date{}

\newtheorem*{theorem}{Proposition 1.2}


\theoremstyle{definition}
\newtheorem{exerciseinner}{\underline{Exercise}}
\newenvironment{exercise}[1]{%
	\renewcommand\theexerciseinner{#1}%
	\exerciseinner
}{\endexerciseinner}

\renewcommand*{\proofname}{Solution}
\renewenvironment{proof}{{\noindent\bfseries\underline{Solution}.}}{\qed}

\makeatletter
\newcounter{proofpart}
\xpretocmd{\proof}{\setcounter{proofpart}{0}}{}{}
\newcommand{\proofpart}[1]{%
	\par
	\addvspace{\medskipamount}%
	\stepcounter{proofpart}%
	\noindent\textbf{{Part \theproofpart: }}#1\par\nobreak\smallskip
	\@afterheading
}
\makeatother



\begin{document}
\maketitle

\section{Compactness}
\begin{exercise}{1.3}
	Show that the property of being a non-bipartite graph is not finitely axiomatizable.
	
	(Hint: use a particular property of bipartite graphs concerning cycles and a standard Compactness
	argument).
\end{exercise}
\begin{proof}
	Let's recall the definition of a bipartite graph: a bipartite graph is a graph whose vertices can be divided into two disjoint and independent sets $U$ and $V$ such that every edge connects a vertex in $U$ to one in $V$. Equivalently, a bipartite graph is a graph that does not contain any odd-length cycles.
	
	To prove that being a non-bipartite graph is not finitely axiomatizable, we're going to exploit the latter definition, and we will basically repeat the proof for the finiteness property that we saw during the lectures.
	
	Let's proceed by contradiction: we assume that being a non-bipartite graph is finitely axiomatizable, and that $T$ is the sentence that axiomatizes it. By definition, this means that, for any structure $\mathfrak{A}$, the following holds:
	
	\[
		\mathfrak{A} \models T \iff \mathfrak{A} \text{ is a non-bipartite graph}.
	\]
	
	We then consider the theory composed by the sentence $T$ augmented by sentences expressing the fact that it doesn't exist an odd-length cycle of length $2n + 1$. Such a sentence is easily expressible in first-order logic with equality:
	
	\[
		A_n := \lnot (\exists x_1 \dots \exists x_{2n + 1}
		((\bigwedge_{1 \le i \le j \le 2n + 1} x_i \ne x_j) \ \land \ (\bigwedge_{1 \le i \le 2n + 1} E(x_i, x_{i + 1})) \ \land \  (E(x_1, x_{2n + 1}))))
	\]
	
	We then consider the theory
	\[
		T \cup \{ A_n \, :\,  n \in \mathbf{N} \}.
	\]
	Consider now a finite subset $X$ of this theory. 
	
	Now $X$ can either contain or not the sentence $T$, along with a finite number of sentences from $\{ A_n \, :\,  n \in \mathbf{N} \}$, say $A_{n_1}, \dots, A_{n_q}$.
	
	Either way, we claim that $X$ has a model.
	
	\proofpart{$X$ contains $T$}
	Let's suppose that $X$ contains $T$ and a set of sentences $A_{n_1}, \dots, A_{n_q}$ from $\{ A_n \, :\,  n \in \mathbf{N} \}$. First, $T$ is true for all non bipartite graphs, which corresponds to the graphs that have at least one odd-length cycle. Second, any structure without a cycle of length $2n_{n_1} + 1, \dots, 2n_{n_q} + 1$ satisfies $A_{n_1}, \dots, A_{n_q}$. Therefore, any structure that doesn't contain any cycles of length $2n_{n_1} + 1, \dots, 2n_{n_q} + 1$ and simultaneously contain at least one odd-length cycle (specifically, an odd cycle different from the ones we just described, that is an odd-length cycle of length $2n_k + 1$ with $k \ne n_{n_1}, \dots, k \ne n_{n_q}$), satisfies $X$.
	
	\proofpart{$X$ doesn't contain $T$}
	The second possibility is that $X$ only contains a set of sentences $A_{n_1}, \dots, A_{n_q}$ from $\{ A_n \, :\,  n \in \mathbf{N} \}$. Any structure without a cycle of length $2n_{n_1} + 1, \dots, 2n_{n_q} + 1$ satisfies $A_{n_1}, \dots, A_{n_q}$. Therefore, any structure that doesn't contain any cycles of length $2n_{n_1} + 1, \dots, 2n_{n_q} + 1$ satisfies $X$.
	
	
	We have thus proved that any finite subset of $T \cup \{ A_n \, :\,  n \in \mathbf{N} \}$ has a model. By the Compactness Theorem, it then follows that the whole theory has a model. Let this model be $\mathfrak{A}$. On the one hand, since $\mathfrak{A} \models T$, it must be the case that $A$ is a non bipartite graph, thus it contains at least one odd-length cycle. On the other, since $\mathfrak{A} \models A_n$ for all $n \in \mathbf{N}$, it must be the case that no odd-length cycle is contained in $\mathfrak{A}$. Therefore, $\mathfrak{A}$ cannot exist. Thus, our hypothesis that $T$ exists is contradictory.
	
	
\end{proof}

\begin{exercise}{1.9}
	Consider the class of undirected graphs with no self-loop.

	A graph is \textbf{acyclic} if, for each $n \ge 3$ it does not contain distinct vertices $x_1, \dots, x_n$ such that $x_i$ is adjacent to $x_{i+1}$ for each $1 \le i < n$ and $x_n$ is adjacent to $x_1$. Prove that the property of being an acyclic graph is
	not finitely axiomatizable in the first-order language of graphs.
	
	(Hint: Use Compactness).
\end{exercise}
\begin{proof}
	We argue by way of contradiction. We suppose that the property of being acyclic is finitely axiomatizable, and that $T$ is a sentence that axiomatizes it. By definition, this means that, for any structure $\mathfrak{A}$, the following holds:
	
	\[
		\mathfrak{A} \models T \iff \mathfrak{A} \text{ is acyclic}.
	\]
	
	Let's consider now the set of sentences expressing the fact that it doesn't exist a cycle of length $n$. Such a sentence is easily expressible in first-order logic with equality:
	
	\[
		A_n := \lnot 
		(
			\exists x_1 \dots \exists x_n
			(
				(\bigwedge_{1 \le i \le j \le n} x_i \ne x_j) \ \land \ (\bigwedge_{1 \le i \le n} E(x_i, x_{i + 1})) \ \land \  (E(x_1, x_{n}))
			)
		)
	\]
	Let's now recall the following proposition from handout 14:
		
	\begin{theorem}[Handout 14]
		The Compactness Theorem is equivalent to the following double implication:
		\[
			T \models E \text{ if and only if for some finite } T_0 \subseteq T \text{ we have } T_0 \models E
		\]
	\end{theorem}
	
	Now, it's trivially true that $\{ A_n : n \ge 3 \} \models T$. Therefore, for proposition 1.2, we know that it exists a finite subset $T_0 \subseteq \{ A_n : n \ge 3 \}$ such that $T_0 \models T$. This is an absurd: let's assume $T_0 = A_{n_1}, \dots, A_{n_q}$. We can always consider a structure $\mathfrak{A}$ such that it doesn't contain any cycle of length $n_{n_1}, \dots, n_{n_q}$, and that simultaneously contains a cycle of length $k = \max \{n_{n_1}, \dots, n_{n_q}\} + 1$. Therefore, we can conclude that the property of being acyclic is not finitely axiomatizable.
	
\end{proof}

\section{Group 2}
\begin{exercise}{2.1}
	Prove that $\omega$-consistency implies consistency.
\end{exercise}
\begin{proof}
	\dots
\end{proof}
\end{document}