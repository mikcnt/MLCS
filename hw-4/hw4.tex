\documentclass[12pt]{article}
\usepackage{graphicx}
\graphicspath{ {./images/} }
\usepackage[utf8]{inputenc}
\usepackage{amsthm,xpatch}
\usepackage{amsmath}
\usepackage{tikz} 
\usepackage{geometry}
\usepackage{amssymb}
\usepackage{enumerate}
\usepackage{float}
\usepackage[english]{babel}
\usetikzlibrary{arrows, automata, calc, positioning}
\geometry{margin=2cm}

\title{MLCS - Homework 4}
\author{Michele Conti \\ \texttt{1599133}}
\date{}

\newtheorem*{theorem}{Proposition 1.2}


\theoremstyle{definition}
\newtheorem{exerciseinner}{\underline{Exercise}}
\newenvironment{exercise}[1]{%
	\renewcommand\theexerciseinner{#1}%
	\exerciseinner
}{\endexerciseinner}

\renewcommand*{\proofname}{Solution}
\renewenvironment{proof}{{\noindent\bfseries\underline{Solution}.}}{\qed}

\makeatletter
\newcounter{proofpart}
\xpretocmd{\proof}{\setcounter{proofpart}{0}}{}{}
\newcommand{\proofpart}[1]{%
	\par
	\addvspace{\medskipamount}%
	\stepcounter{proofpart}%
	\noindent\textbf{{Part \theproofpart: }}#1\par\nobreak\smallskip
	\@afterheading
}
\makeatother



\begin{document}
\maketitle

\section{Compactness}
\begin{exercise}{1.3}
	Show that the property of being a non-bipartite graph is not finitely axiomatizable.
	
	(Hint: use a particular property of bipartite graphs concerning cycles and a standard Compactness
	argument).
\end{exercise}
\begin{proof}
	Let's recall the definition of a bipartite graph: a bipartite graph is a graph whose vertices can be divided into two disjoint and independent sets $U$ and $V$ such that every edge connects a vertex in $U$ to one in $V$. Equivalently, a bipartite graph is a graph that does not contain any odd-length cycles.
	
	To prove that being a non-bipartite graph is not finitely axiomatizable, we're going to exploit the latter definition, and we will basically repeat the proof for the finiteness property that we saw during the lectures.
	
	Let's proceed by contradiction: we assume that being a non-bipartite graph is finitely axiomatizable, and that $T$ is the sentence that axiomatizes it. By definition, this means that, for any structure $\mathfrak{A}$, the following holds:
	
	\[
		\mathfrak{A} \models T \iff \mathfrak{A} \text{ is a non-bipartite graph}.
	\]
	
	We then consider the theory composed by the sentence $T$ augmented by sentences expressing the fact that it doesn't exist an odd-length cycle of length $2n + 1$. Such a sentence is easily expressible in first-order logic with equality:
	
	\[
		A_n := \lnot (\exists x_1 \dots \exists x_{2n + 1}
		((\bigwedge_{1 \le i \le j \le 2n + 1} x_i \ne x_j) \ \land \ (\bigwedge_{1 \le i \le 2n + 1} E(x_i, x_{i + 1})) \ \land \  (E(x_1, x_{2n + 1}))))
	\]
	
	We then consider the theory
	\[
		T \cup \{ A_n \, :\,  n \in \mathbf{N} \}.
	\]
	Consider now a finite subset $X$ of this theory. 
	
	Now $X$ can either contain or not the sentence $T$, along with a finite number of sentences from $\{ A_n \, :\,  n \in \mathbf{N} \}$, say $A_{n_1}, \dots, A_{n_q}$.
	
	Either way, we claim that $X$ has a model.
	
	\proofpart{$X$ contains $T$}
	Let's suppose that $X$ contains $T$ and a set of sentences $A_{n_1}, \dots, A_{n_q}$ from $\{ A_n \, :\,  n \in \mathbf{N} \}$. First, $T$ is true for all non bipartite graphs, which corresponds to the graphs that have at least one odd-length cycle. Second, any structure without a cycle of length $2n_{n_1} + 1, \dots, 2n_{n_q} + 1$ satisfies $A_{n_1}, \dots, A_{n_q}$. Therefore, any structure that doesn't contain any cycles of length $2n_{n_1} + 1, \dots, 2n_{n_q} + 1$ and simultaneously contain at least one odd-length cycle (specifically, an odd cycle different from the ones we just described, that is an odd-length cycle of length $2n_k + 1$ with $k \ne n_{n_1}, \dots, k \ne n_{n_q}$), satisfies $X$.
	
	\proofpart{$X$ doesn't contain $T$}
	The second possibility is that $X$ only contains a set of sentences $A_{n_1}, \dots, A_{n_q}$ from $\{ A_n \, :\,  n \in \mathbf{N} \}$. Any structure without a cycle of length $2n_{n_1} + 1, \dots, 2n_{n_q} + 1$ satisfies $A_{n_1}, \dots, A_{n_q}$. Therefore, any structure that doesn't contain any cycles of length $2n_{n_1} + 1, \dots, 2n_{n_q} + 1$ satisfies $X$.
	
	
	We have thus proved that any finite subset of $T \cup \{ A_n \, :\,  n \in \mathbf{N} \}$ has a model. By the Compactness Theorem, it then follows that the whole theory has a model. Let this model be $\mathfrak{A}$. On the one hand, since $\mathfrak{A} \models T$, it must be the case that $A$ is a non bipartite graph, thus it contains at least one odd-length cycle. On the other, since $\mathfrak{A} \models A_n$ for all $n \in \mathbf{N}$, it must be the case that no odd-length cycle is contained in $\mathfrak{A}$. Therefore, $\mathfrak{A}$ cannot exist. Thus, our hypothesis that $T$ exists is contradictory.
	
	
\end{proof}

\begin{exercise}{1.5}
	Prove the following: If a property $P$ and its complement are axiomatizable then $P$ is finitely axiomatizable.
\end{exercise}

\begin{proof}
	Assume that $P$ is axiomatized by a set of sentences $U$, and that $\lnot P$ is axiomatized by a set of sentences $V$. Formally:
	
	\[
		\mathfrak{U} \models U \iff \mathfrak{U} \text{ has the property } P
	\]
	and
	\[
		\mathfrak{V} \models V \iff \mathfrak{V} \text{ has the property } \lnot P.
	\]
	
	Now, if we consider the set of sentences $U \cup V$, that is both the sentences that axiomatize $P$ and the ones that axiomatize $\lnot P$, these don't have a model.
	
	Therefore, according to the Compactness Theorem, there exists a finite subset $W \subseteq U \cup V$ such that $W$ has no model. We can now assume that $W$ is made by a subset of sentences from $U$, say $U_0 = \{u_0, \dots, u_p\}$, and a set of sentences from $V$, say $V_0 = \{v_0, \dots, v_q\}$:
	
	\[
		W = U_0 \cup V_0 = \{u_0, \dots, u_p, v_0, \dots, v_q\}.
	\]
	
	We can in fact rule out all other possibilities, which basically consist in the fact that either $U_0$ or $V_0$ are empty. For example, if $V_0 = \emptyset$, then we would have that $W = U_0$ has no model. But this can not be the case, since we know by hypothesis that $U$ has a model, and that therefore (Compactness Theorem) that every finite subset (including $U_0$) has also a model. The same reasoning can be applied to exclude the possibility that $U_0 = \emptyset$.
	
	Now, for the compactness theorem, since $U$ has a model, every finite subset of $U$ should have a model too, including $U_0$. By the same reasoning, $V_0$ should have a model too. Let this models be respectively $\mathfrak{U}_0$ and $\mathfrak{V}_0$.
	
	\textbf{Notation detail}: When we write that a set of sentences $A$ has a model $\mathfrak{A}$, we mean that $\mathfrak{A}$ is the class of structures that satisfy the sentences $A$. This way, we can refer to the union and intersection of two models in the following way:
	
	\begin{itemize}
		\item Given two set of sentences $A$ and $B$ and their respective models $\mathfrak{A}$ and $\mathfrak{B}$, we can define the union between the two models $\mathfrak{A} \cup \mathfrak{B}$ as the class of structures that either satisfy $A$ or $B$.
		\item Given two set of sentences $A$ and $B$ and their respective models $\mathfrak{A}$ and $\mathfrak{B}$, we can define the union between the two models $\mathfrak{A} \cup \mathfrak{B}$ as the class of structures that both satisfy $A$ and $B$.
	\end{itemize}
	
	Now we know that the models $\mathfrak{U}$ and $\mathfrak{V}$ are disjoint, since there is no structure that both satisfy $P$ and $\lnot P$. Same goes for $\mathfrak{U}_0$ and $\mathfrak{V}_0$. Moreover, we know that $\mathfrak{U} \subseteq \mathfrak{U_0}$ and that $\mathfrak{V} \subseteq \mathfrak{V_0}$, since adding new sentences acts like the creation of new constraints, and therefore decreases the size of the space of the represented structures.
	
	Finally, $\mathfrak{U} \cup \mathfrak{V}$ is the class of all models, since a structure either satisfy $P$ or $\lnot P$, and therefore either belongs to $\mathfrak{U}$ or $\mathfrak{V}$. Knowing that $\mathfrak{U} \subseteq \mathfrak{U}_0$ and $\mathfrak{V} \subseteq \mathfrak{V}_0$, we can deduce that $\mathfrak{U} = \mathfrak{U}_0$, and thus that $P$ is finitely axiomatizable (i.e., by the sentences in $U_0$).
	
\end{proof}

\section{Group 2}
\begin{exercise}{2.1}
	Prove that $\omega$-consistency implies consistency.
\end{exercise}
\begin{proof}
	Let's first recall the definitions of consistency and $\omega$-consistency:
	\begin{itemize}
		\item A theory $T$ is consistent if there is no formula $F$ such that $\vdash F$ and $\vdash \lnot F$. Equivalently, $T$ is consistent if $\vdash F$ for every formula $F$.
		\item A theory $T$ is $\omega$-consistent if there is no formula $A(x)$ such that for each $n \in \mathbf{N}$ we have both $T \vdash A(n)$ and $T \vdash \lnot (\forall x) A(x)$.
	\end{itemize}
	
	To prove that $\omega$-consistency implies consistency, let's consider an $\omega$-consistent theory $T$ and the formula $A(x): x = x$. Let's now proceed by contradiction: let's assume that $T$ is not consistent. By definition, since $T$ is not consistent we have that $\vdash \exists \, x\,  A(x)$, since all formulas are true. But this can not be the case, since $T$ is $\omega$-consistent, and therefore it can't be the case that $T \vdash \lnot (\forall x) A(x)$. Contradiction.
\end{proof}
\end{document}