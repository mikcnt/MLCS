\documentclass[12pt]{article}
\usepackage{graphicx}
\graphicspath{ {./images/} }
\usepackage[utf8]{inputenc}
\usepackage{amsthm,xpatch}
\usepackage{amsmath}

\usepackage{geometry}
\usepackage{amssymb}
\usepackage{enumerate}
\geometry{margin=2cm}

\title{MLCS - Homework 2}
\author{Michele Conti \\ \texttt{1599133}}
\date{}


\theoremstyle{definition}
\newtheorem{exerciseinner}{Exercise}
\newenvironment{exercise}[1]{%
	\renewcommand\theexerciseinner{#1}%
	\exerciseinner
}{\endexerciseinner}

\renewcommand*{\proofname}{Solution}

\makeatletter
\newcounter{proofpart}
\xpretocmd{\proof}{\setcounter{proofpart}{0}}{}{}
\newcommand{\proofpart}[1]{%
	\par
	\addvspace{\medskipamount}%
	\stepcounter{proofpart}%
	\noindent\emph{Part \theproofpart: #1}\par\nobreak\smallskip
	\@afterheading
}
\makeatother

\begin{document}
\maketitle

\begin{exercise}{1.2}
	Consider the following two structures $\mathfrak{G}_1$ and $\mathfrak{G}_2$ for the language of graphs: Write at least two sentences distinguishing the two structures. Discuss the EF-game played on these structures: for what $k$ can the Duplicator win the $k$-rounds game? For what $k$ can Spoiler win?
	\begin{figure}[h]
		\includegraphics[width=8cm]{ex1.2}
		\centering
	\end{figure}
\end{exercise}
\end{document}