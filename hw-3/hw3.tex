\documentclass[12pt]{article}
\usepackage{graphicx}
\graphicspath{ {./images/} }
\usepackage[utf8]{inputenc}
\usepackage{amsthm,xpatch}
\usepackage{amsmath}
\usepackage{tikz} 
\usepackage{geometry}
\usepackage{amssymb}
\usepackage{enumerate}
\usepackage{float}
\usepackage[english]{babel}
\usetikzlibrary{arrows, automata, calc, positioning}
\geometry{margin=2cm}

\title{MLCS - Homework 3}
\author{Michele Conti \\ \texttt{1599133}}
\date{}

\theoremstyle{definition}
\newtheorem{exerciseinner}{\underline{Exercise}}
\newenvironment{exercise}[1]{%
	\renewcommand\theexerciseinner{#1}%
	\exerciseinner
}{\endexerciseinner}

\renewcommand*{\proofname}{Solution}
\renewenvironment{proof}{{\noindent\bfseries\underline{Solution}.}}{\qed}

\makeatletter
\newcounter{proofpart}
\xpretocmd{\proof}{\setcounter{proofpart}{0}}{}{}
\newcommand{\proofpart}[1]{%
	\par
	\addvspace{\medskipamount}%
	\stepcounter{proofpart}%
	\noindent\textbf{{Part \theproofpart: }}#1\par\nobreak\smallskip
	\@afterheading
}
\makeatother



\begin{document}
\maketitle

\section{Computable, c.e. and not computable Problems}
\begin{exercise}{1.2}
	Show by informal arguments the following points.
	\begin{enumerate}[\quad{(}1{)}]
		\item If $L_1 \cap L_2$ is not computable and $L_2$ is computable then $L_1$ is not computable.
		\item If $L_1 \cup L_2$ is not computable and $L_2$ is computable then $L_1$ is not computable.
		\item If $L_1 \setminus L_2$ is not computable and $L_2$ is computable then $L_1$ is not computable.
		\item If $L_2 \setminus L_1$ is not computable and $L_2$ is computable then $L_1$ is not computable.
		\item If $L_1$ and $L_2$ are computably enumerable then $L_1 \cup L_2$ and $L_1 \cap L_2$ are computably enumerable.
	\end{enumerate}
\end{exercise}
\begin{proof}
The first four points can be proved proceeding by contradiction, slightly tweeking the rationale at each case.

\proofpart{}
Let's assume by contradiction that $L_1$ is computable. By hypothesis, we also know that $L_2$ is computable, therefore there exist two Turing Acceptors $M_1$ and $M_2$ such that, for $i = 1, 2$:
\[
	\text{if } x \in L_i \text{ then } M_i \text{ accepts } x; \text{ if } x \not\in L_i \text{ then } M_i \text{ rejects } x.
\]

Let's consider now a new Turing Acceptor $M$, defined as follows:

\[
	M(x)= \begin{cases}
		M_1(x) & \text{if } x \in L_1 \cap L_2 \\
		M_1(x) & \text{if } x \in L_2 \setminus L_1 \\
		M_2(x) & \text{if } x \in L_1 \setminus L_2 \\
		M_2(x) & \text{if } x \not\in L_1 \cup L_2
	\end{cases}
\]
where $M_1(x)$ and $M_2(x)$ are respectively the results of the computation of the Turing machines $M_1$ and $M_2$ on the element $x$.

The existence of such machine is, by definition, equivalent to the fact that $L_1 \cap L_2$ is decidable, since the elements in $L_1 \cap L_2$ are going to be accepted by the machine $M$ and all the elements outside $L_1 \cap L_2$ are going to be rejected. This is absurd by hypothesis. Therefore, we can conclude that $L_1$ is not computable.


Notice also that the for the first and last cases in the definition of $M$, we could have chosen either $M_1$ and $M_2$, getting the same results.

\proofpart{}
Similarly to the previous point, we can proceed by contradiction, assuming that $L_1$ is computable, and considering a new Turing Acceptor $M$:

\[
	M(x)= \begin{cases}
		M_1(x) & \text{if } x \in L_1 \\
		M_2(x) & \text{if } x \in L_2 \\
		M_2(x) & \text{if } x \not\in L_1 \cup L_2.
	\end{cases}
\]

The existence of this machine would be equivalent to the fact that $L_1 \cup L_2$ is computable, which again would be absurd. Therefore, we can conclude that $L_1$ is not computable.

\proofpart{}
Similarly to the previous points, we can proceed by contradiction, assuming that $L_1$ is computable. Again, we'll consider a new Turing Acceptor $M$, but we'll proceed in a slightly different way.

Since $L_1$ and $L_2$ are computable, given a point $x$ they either accept or reject it. Therefore, we can define the new Turing Acceptor based on the behaviour of the two Acceptors:

\[
\begin{cases}
M(x) \uparrow & \text{if } M_1(x) \uparrow \text{ and }  M_2(x) \downarrow \\
M(x) \downarrow & \text{otherwise}.
\end{cases}
\]

This machine would accept every element in $L_1 \setminus L_2$, and would reject everything else, therefore its existence would be equivalent to the fact that $L_1 \setminus L_2$ is computable. This is absurd, and thus we can conclude that $L_1$ is not computable.

\proofpart{}
We can proceed in a specular way to the previous point. Proceeding by contradiction, we assume that $L_1$ is computable. Let's consider the new Turing Acceptor $M$:

\[
\begin{cases}
M(x) \uparrow & \text{if } M_1(x) \downarrow \text{ and }  M_2(x) \uparrow \\
M(x) \downarrow & \text{otherwise}.
\end{cases}
\]

This machine would accept every element in $L_2 \setminus L_1$, and would reject everything else, therefore its existence would be equivalent to the fact that $L_2 \setminus L_1$ is computable. This is absurd, and thus we can conclude that $L_1$ is not computable.

\proofpart{}
For this last point, we'll have to change our strategy. Both implications ($L_1 \cup L_2$ is computable and $L_1 \cap L_2$ is computable) can be proved using similar reasonings, but we'll analyze both of them separately.

Let's start by proving that $L_1 \cup L_2$ is computably enumerable. Consider the two Acceptors $M_1$ and $M_2$ and a point $x$. Now, if we choose one of the two machines, we can express the computation of $x$ as a collection of snapshots, and therefore we can analyze the whole computation step by step. Let's now consider this mechanism for both machines simultaneously: we first compute one computation step of the Acceptor $M_1$, then we compute one computation step of the Acceptor $M_2$, and we cyclically continue using this strategy.

There are now two cases. If $x \not\in L_1 \cup L_2$, then both machines would diverge, as both $L_1$ and $L_2$ are computably enumerable. On the other hand, if $x \in L_1 \cup L_2$, one of the two machines would eventually accept, and we can therefore stop the computation. This proves that $L_1 \cup L_2$ is computably enumerable.

Now to the second implication: $L_1 \cap L_2$ is computably enumerable. Following the same idea, we can again analyze the machines behaviour in a step-by-step fashion. Now, if $x \not\in L_1 \cup L_2$, both machines would still diverge. On the other hand, we would stop the computation only if both the machines accepts at a certain time step. This would clearly never happen if $x \in L_1 \setminus L_2$ or if $x \in L_2 \setminus L_1$ (since one of the two machines would continue the computation forever), and would certainly happen on the intersection $L_1 \cap L_2$, since both machines are computably enumerable. This proves that $L_1 \cap L_2$ is computably enumerable.

\end{proof}
\section{\textit{NP} problems}

\end{document}
