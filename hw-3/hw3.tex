\documentclass[12pt]{article}
\usepackage{graphicx}
\graphicspath{ {./images/} }
\usepackage[utf8]{inputenc}
\usepackage{amsthm,xpatch}
\usepackage{amsmath}
\usepackage{tikz} 
\usepackage{geometry}
\usepackage{amssymb}
\usepackage{enumerate}
\usepackage{float}
\usepackage[english]{babel}
\usetikzlibrary{arrows, automata, calc, positioning}
\geometry{margin=2cm}

\title{MLCS - Homework 3}
\author{Michele Conti \\ \texttt{1599133}}
\date{}

\theoremstyle{definition}
\newtheorem{exerciseinner}{\underline{Exercise}}
\newenvironment{exercise}[1]{%
	\renewcommand\theexerciseinner{#1}%
	\exerciseinner
}{\endexerciseinner}

\renewcommand*{\proofname}{Solution}
\renewenvironment{proof}{{\noindent\bfseries\underline{Solution}.}}{\qed}

\makeatletter
\newcounter{proofpart}
\xpretocmd{\proof}{\setcounter{proofpart}{0}}{}{}
\newcommand{\proofpart}[1]{%
	\par
	\addvspace{\medskipamount}%
	\stepcounter{proofpart}%
	\noindent\textbf{{Part \theproofpart: }}#1\par\nobreak\smallskip
	\@afterheading
}
\makeatother



\begin{document}
\maketitle

\section{Computable, c.e. and not computable Problems}
\begin{exercise}{1.2}
	Show by informal arguments the following points.
	\begin{enumerate}[\quad{(}1{)}]
		\item If $L_1 \cap L_2$ is not computable and $L_2$ is computable then $L_1$ is not computable.
		\item If $L_1 \cup L_2$ is not computable and $L_2$ is computable then $L_1$ is not computable.
		\item If $L_1 \setminus L_2$ is not computable and $L_2$ is computable then $L_1$ is not computable.
		\item If $L_2 \setminus L_1$ is not computable and $L_2$ is computable then $L_1$ is not computable.
		\item If $L_1$ and $L_2$ are computably enumerable then $L_1 \cup L_2$ and $L_1 \cap L_2$ are computably enumerable.
	\end{enumerate}
\end{exercise}
\begin{proof}
The first four points can be proved proceeding by contradiction, slightly tweeking the rationale at each case.

\proofpart{}
Let's assume by contradiction that $L_1$ is computable. By hypothesis, we also know that $L_2$ is computable, therefore there exist two Turing Acceptors $M_1$ and $M_2$ such that, for $i = 1, 2$:
\[
	\text{if } x \in L_i \text{ then } M_i \text{ accepts } x; \text{ if } x \not\in L_i \text{ then } M_i \text{ rejects } x.
\]

Let's consider now a new Turing Acceptor $M$, defined as follows:

\[
	M(x)= \begin{cases}
		M_1(x) & \text{if } x \in L_1 \cap L_2 \\
		M_1(x) & \text{if } x \in L_2 \setminus L_1 \\
		M_2(x) & \text{if } x \in L_1 \setminus L_2 \\
		M_2(x) & \text{if } x \not\in L_1 \cup L_2
	\end{cases}
\]
where $M_1(x)$ and $M_2(x)$ are respectively the results of the computation of the Turing machines $M_1$ and $M_2$ on the element $x$.

The existence of such machine is, by definition, equivalent to the fact that $L_1 \cap L_2$ is decidable, since the elements in $L_1 \cap L_2$ are going to be accepted by the machine $M$ and all the elements outside $L_1 \cap L_2$ are going to be rejected. This is absurd by hypothesis. Therefore, we can conclude that $L_1$ is not computable.


Notice also that the for the first and last cases in the definition of $M$, we could have chosen either $M_1$ and $M_2$, getting the same results.

\proofpart{}
Similarly to the previous point, we can proceed by contradiction, assuming that $L_1$ is computable, and considering a new Turing Acceptor $M$:

\[
	M(x)= \begin{cases}
		M_1(x) & \text{if } x \in L_1 \\
		M_2(x) & \text{if } x \in L_2 \\
		M_2(x) & \text{if } x \not\in L_1 \cup L_2
	\end{cases}
\]

The existence of this machine would be equivalent to the fact that $L_1 \cup L_2$ is computable, which again would be absurd. Therefore, we can conclude that $L_1$ is not computable.

\end{proof}
\section{\textit{NP} problems}

\end{document}
